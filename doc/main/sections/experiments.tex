\section{Eksperymenty}
Wykonaliśmy szereg testów, sprawdzających jak działa aplikacja w zależności od następujących parametrów:
\begin{itemize}
    \item liczebność populacji
    \item liczebność populacji $\mu$
    \item liczba prostokątów w osobniku
    \item strategia krzyżowania
    \item strategia wybierania kolejnej generacji populacji
\end{itemize}
\subsection{Eksperymenty i wyniki}
W celu uzyskania możliwie jak najbardziej prawdziwych wyników dla każdego przypadku testowego, każdy eksperyment z danymi parametrami był przeprowadzony 5 razy. Wszystkie testy zostały wykonane dla obrazu {\it simple\_img.jpg} oraz dla z warunkami zakończenia 1000 iteracji lub dokładność 99\%. Gdy nie było to przedmiotem eksperymentu to metodą krzyżowania było uśrednianie, a metodą wybierania - wybór najlepszych.
\subsection*{Wpływ liczebności populacji}
Test wpływu liczebności populacji został przeprowadzony dla populacji o wielkościach: 2, 10, 15, 30, 40, 50, 100, liczbie prostokątów w osobniku równej 20, wielkości populacji $\mu$ będącej 1,5 raza większej od wielkości testowanej populacji.
\subsection*{Wpływ liczebności populacji $\mu$}
Test wpływu liczebności populacji $\mu$ został przeprowadzony dla stałej wielkości populacji $\lambda$ równej 40 na podstawie której kolejne przypadki testowe był obliczane za pomocą wartości procentowych: 110\%(44), 130\%(52),
150\%(60), 200\%(80), 250\%(100), 300\%(120). Liczba prostokątów w osobniku była równa 20.
\subsection*{Wpływ liczby prostokątów w osobniku}
Następny test został przeprowadzony dla następujących liczb prostokątów w osobniku: 10, 20, 50, 100, 200, 300. Wielkość populacji wynosiła 10, podpopulacji 15. Pozostałe parametry zostały niezmienione względem poprzedniego testu. 
\subsection*{Wpływ użytej strategii krzyżowania}

\subsection*{Wpływ użytej strategii wybierania kolejnej populacji}
\subsection{Wnioski}