\documentclass{article}
\usepackage[T1]{fontenc}
\usepackage[polish]{babel}
\usepackage[utf8]{inputenc}

\title{Ewolucyjny malarz \\ \large Cel projektu}
\author{Rafał Kwiatkowski, Franciszek Sioma}


\begin{document}
\pagenumbering{gobble}
\maketitle

Celem projektu {\it Ewolucyjny Malarz} jest stworzenie aplikacji wykorzystującej algorytm ewolucyjny, która na wejściu przyjmuje kolorowy obrazek, a na wyjściu zwraca możliwe jak najdokładniejsze odwzorowanie tego obrazu przy użyciu kwadratów RGBA. 

W algorytmie ewolucyjnym istotne jest to, aby zdefiniować czym będzie {\it Osobnik}. W naszym przypadku osobnikiem będzie {\it n} prostokątów RGBA. Każdy prostokąt będzie się składał z 8 wartości oprócz: 
\begin{description}
    \item[R] kanał czerwony
    \item[G] kanał zielony
    \item[B] kanał niebieski
    \item[A] kanał alpha (mówi o przezroczystości danego piksela)
    \item[X] pozycja względem osi poziomej
    \item[Y] pozycja względem osi pionowej
    \item[W] szerokość
    \item[H] wysokość       
\end{description}

Kolejnym niejasnym punktem może być sposób obliczania funkcji przystosowania ({\it J}). W naszym przypadku funkcja ta będzie przyjmować wartości od 0 do 1 i będzie obliczana na podstawie porównywania piksel po pikselu.

Planujemy wykonać szereg testów, które pomogą nam odpowiedzieć na pytanie jak działa aplikacja w zależności od następujących parametrów:
\begin{itemize}
    \item liczebność populacji
    \item liczba prostokątów w danym osobniku
    \item strategia krzyżowania
    \item strategia mutowania
    \item algorytm ewolucyjny
\end{itemize}
\end{document}