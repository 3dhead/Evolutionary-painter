\documentclass{article}
\usepackage[T1]{fontenc}
\usepackage[polish]{babel}
\usepackage[utf8]{inputenc}

\title{Ewolucyjny malarz }%\\ \large Cel projektu}
\author{Rafał Kwiatkowski, Franciszek Sioma}


\begin{document}
\pagenumbering{gobble}
\maketitle

\section{Opis projektu}
\subsection{Cel projektu}
Celem projektu {\it Ewolucyjny Malarz} było stworzenie aplikacji wykorzystującej algorytm ewolucyjny. Aplikacja przyjmuje na wejściu kolorowy obrazek, a na wyjściu zwraca możliwe jak najdokładniejsze odwzorowanie obrazu wejściowego przy użyciu prostokątów RGBA. 

\subsection{Przyjęte założenia}
W algorytmie ewolucyjnym istotne jest to, aby zdefiniować czym będzie {\it Osobnik}. W naszym przypadku osobnikiem jest obraz stworzony z {\it n} prostokątów RGBA. Każdy prostokąt składa się z 8 wartości: 
\begin{description}
    \item[R] kanał czerwony
    \item[G] kanał zielony
    \item[B] kanał niebieski
    \item[A] kanał alpha (mówi o przezroczystości danego piksela)
    \item[X] pozycja względem osi poziomej
    \item[Y] pozycja względem osi pionowej
    \item[W] szerokość
    \item[H] wysokość       
\end{description}

Kolejnym niejasnym punktem może być sposób obliczania funkcji przystosowania ({\it J}). W naszym przypadku funkcja ta będzie przyjmować wartości od 0 do 1 i będzie obliczana na podstawie porównywania piksel po pikselu obrazu stworzonego przez osobnika z obrazem oryginalnym.
\subsection{Wkład autorów}
\begin{itemize}
    \item Obsługa obrazów - Rafał Kwiatkowski
    \item Algorytm Ewolucyjny - Franciszek Sioma
    \item Testy i eksperymenty - Rafał Kwiatkowski
    \item Dokumentacja - Franciszek Sioma
\end{itemize}
\subsection{Decyzje projektowe}
Algorytm zakłada dwa rodzaje krzyżowania osobników:
\begin{itemize}
    \item uśrednianie
    \item interpolacja
\end{itemize}
W naszym programie postanowiliśmy umieścić oba te sposoby, by móc przebadać ich wpływ na końcowy wynik algorytmu.
\subsection{Wykorzystane narzędzia i biblioteki}
Do napisania aplikacji użyliśmy języka Python w wersji: 3.8, dokumentacja została stworzona przy użyciu języka Latex, a IDE z którego korzystaliśmy to Visual Studio Code. Użyliśmy również systemu kontroli wersji Git.
Spis użytych bibliotek znajduje się w pliku {\it requirements.txt}.

\section{Testy}
Planujemy wykonać szereg testów, które pomogą nam odpowiedzieć na pytanie jak działa aplikacja w zależności od następujących parametrów:
\begin{itemize}
    \item liczebność populacji
    \item liczba prostokątów w danym osobniku
    \item strategia krzyżowania
    \item strategia mutowania
\end{itemize}
\subsection{Przeprowadzone testy}
\subsection{Wyniki}
\end{document}